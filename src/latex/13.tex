
\documentclass{beamer}

\newtheorem{myrule}{Rule}

\def\beq{\begin{eqnarray}}
\def\eeq{\end{eqnarray}}
\def\2{\frac{1}{2}}



\mode<presentation>
{
 \usetheme{Warsaw}
%\usecolortheme[rgb={0,0.7,0.1}]{structure}
%  \usetheme{default}
%  \usetheme{Boadilla}
%  \usetheme{Madrid}
%  \usetheme{Pittsburg}
%  \usetheme{Boadilla}

  % or ...


  \setbeamercovered{transparent}
  % or whatever (possibly just delete it)
}


\usepackage{epsfig}
\newtheorem{axiom}{Axiom}
\usepackage[english]{babel}
\usepackage[latin1]{inputenc}
\usepackage{times}
\usepackage[T1]{fontenc}
% Or whatever. Note that the encoding and the font should match. If T1
% does not look nice, try deleting the line with the fontenc.

\title[Game Theory...] % (optional, use only with long paper titles)
{Game Theory - Decision Formulation for Management}


\author[] % (optional, use only with lots of authors)
{MB}
% - Give the names in the same order as the appear in the paper.
% - Use the \inst{?} command only if the authors have different
%   affiliation.

\institute[Oslo University College] % (optional, but mostly needed)
{
  Department of Computer Science\\
  Oslo Univeristy College
% - Use the \inst command only if there are several affiliations.
% - Keep it simple, no one is interested in your street address.
}

\date[MS007A] % (optional, should be abbreviation of conference name)
{November 2006}

\subject{Game Theory - Decision Formulation for Management}
% This is only inserted into the PDF information catalog. Can be left
% out. 
%\usecolortheme{beetle}

\begin{document}

\begin{frame}
  \titlepage
\end{frame}


\begin{frame}
\frametitle{Zero-sum games}

\begin{itemize}
\item Solved by saddlepoint, minimax theorem (von Neumann).

\item Take the payoff matrix for any one of the players (e.g. $\Pi_1)$.
We compare:

\beq
\max_1\min_2 \Pi_1 &\equiv& v_1\\
\min_2\max_1 \Pi_1 &\equiv& v_2
\eeq

\item If $v_1 = v_2$ there is a pure strategy solution, i.e.
a ``best'' answer. 

\item Else there is a mixture of strategies for
both players that yields an equilibrium.

\end{itemize}

\end{frame}


\begin{frame}
\frametitle{Example - zero sum}

Let
$$
\Pi_1 = - \Pi_2 = \left(
\begin{array}{ccc}
1 & -3 & -2\\
2 & 5 & 4\\
2 & 3 & 2\\
\end{array}
\right)
$$
$$
\max_\updownarrow\min_\leftrightarrow \Pi_1 = \max_\updownarrow \left(
\begin{array}{c}
-3\\
2\\
2\\
\end{array}
\right) = 2
$$

$$
\min_\leftrightarrow \max_\updownarrow\Pi_1 = \min_\leftrightarrow (2,5,4) = 2
$$
Thus $v_1=v_2=2$ and the relevant strategies are: 
$$(\updownarrow,\leftrightarrow) =(2,1) or (3,1)$$
\end{frame}



\begin{frame}
\frametitle{Garbage collection game}
\begin{itemize}
\item Users want to keep everything. 
\item Systems need to forget data!
\end{itemize}


\begin{center}
\small
\begin{tabular}{|c|c|c|c|c|}
\hline
Users/System & Ask to tidy & Tidy by date & Tidy above & Quotas\\
             &             &              & Threshold  &       \\
\hline
Tidy when asked & $\pi(1,1)$ & $\pi(1,2)$ & $\pi(1,3)$ & $\pi(1,4)$\\
\hline
Never tidy & $\pi(2,1)$ & $\pi(2,2)$ & $\pi(2,3)$ & $\pi(2,4)$\\
\hline
Conceal files & $\pi(3,1)$ & $\pi(3,2)$ & $\pi(3,3)$ & $\pi(3,4)$\\
\hline
Change timestamps & $\pi(4,1)$ & $\pi(4,2)$ & $\pi(4,3)$ & $\pi(4,4)$\\
\hline
\end{tabular}
\end{center}

Payoff -- mixture of bytes and social satisfaction...
\begin{eqnarray}
\pi = \pi_r({\rm resources}) + \pi_s({\rm satisfaction}).
\end{eqnarray}


\end{frame}



\begin{frame}
\frametitle{Dominated strategies}

Sometimes the payoff matrix can be simplified 
for both players if one
of the players has a weak strategy, i.e. one that there is no reason to play
because it performs worse than all others.

e.g.

$$
\left(
\begin{array}{ccc}
50 & -10 & -20\\
20 & 40 & -44\\
0 & -10 & -40
\end{array}
\right)
\rightarrow
\left(
\begin{array}{cc}
50 & -10 \\
20 & 40 \\
0 & -10 
\end{array}
\right)
$$
\end{frame}

\begin{frame}
\frametitle{Nash equilibrium}

\begin{itemize}
\item Generalization of minimax for non-zero sum games (John Nash).\vspace{0.3cm}

\item Most popular solution concept in game theory. \vspace{0.3cm}

\item Equivalent to minimax for zero-sum games.\vspace{0.3cm}

\item Finds optimal mixed strategies for players in arbitrary games.\vspace{0.3cm}

\item e.g. use GAMBIT software to search numerically 
\begin{itemize}
\item \alert{(Is GAMBIT broken in 2005/6?)}
\end{itemize}
\end{itemize}

\end{frame}



\begin{frame}
\frametitle{Cooperative games}

\begin{itemize}
\item Zero sum games are anti-cooperative.

\item If players (agents) can win by working together in the long-term, we call games
cooperative, e.g. sharing wireless bandwidth in a friendly way or competing
aggressively.

$$
\Pi = \left( 
\begin{tabular}{c|c|c}
   & Compete & Share\\
\hline
Compete & ? & ?\\
Share & ? & ?
\end{tabular}
\right)
$$

\item Does it pay agents to share or to compete? e.g.


\item We need to model payoffs to make these into games.
\end{itemize}
\end{frame}


\begin{frame}
\frametitle{Dilemma games}

What do we do if we have this? Let `C' mean cooperate and `D'
mean ``defect'' from cooperation, i.e. stab in the back!

\psfig{file=dilemma.eps,width=3cm}

\begin{itemize}
\item $R$ is the reward for cooperation.
\item $T$ is the temptation to ``defect''.
\item $S$ is the sucker's payoff.
\item $P$ is the punishment for both losing the trust gamble.
\end{itemize}
\end{frame}


\begin{frame}
\frametitle{Dilemma games II}

\begin{itemize}
\item Criterion $T>R>P>S$. 

\item Here we see both players gamble on each others' kindness.

\begin{itemize}
\item If both cooperate, both get maximum reward.
\item If one cooperates and the other defects, the defector wins by exploiting the other.
\item If both try to exploit each other, they both lose with a minimum payoff.
\end{itemize}

Applications:
\begin{itemize}
\item \alert{Imagine customers sharing an Ethernet (see book).}
\item \alert{Imagine users sharing a fixed size disk.}
\end{itemize}
\end{itemize}
\end{frame}


\begin{frame}
\frametitle{Iterated dilemmas - trade}

Basis for commerce:
\begin{itemize}
\item Treat each game as a single interaction and play the game many times,

\item Each player can only see {\em previous} rounds.

\item The strategy of
`tit for tat' is  found to be VERY effective at stabilizing a players' returns.

\item This is used to argue for the significance of strategy in conflict/cooperation
scenarios.

\item \alert{Shows that there is no incentive to cooperate if agents meet only once} (think
ubiquitous computing).
\end{itemize}

\end{frame}



\begin{frame}
\frametitle{Contract Agreement - Principle Agent Model}


Finding equilibrium contract:\vspace{0.3cm}
\begin{itemize}
\item Emphasizes the order of players and information available.\vspace{0.3cm}

\item Playing for profit - maximize ``profit - cost''.\vspace{0.3cm}

\item Player 1: strategies  -- choose prices.\vspace{0.3cm}

\item Player 2: strategies -- choose expenses.\vspace{0.3cm}
\end{itemize}

\alert{See notes on micro-economic modelling}
\end{frame}



\begin{frame}
\frametitle{Other games...}

Management scenarios that are games?\vspace{0.3cm}
\begin{itemize}
\item Any resource sharing problem\vspace{0.3cm}

\begin{itemize}
\item Load sharing (clusters/server farms)
\item Service Level contracts
\end{itemize}

\item Pricing of services - contracts/outsourcing.\vspace{0.3cm}

\item Finding mixture of services to balance a competitor.\vspace{0.3cm}

\item Find best way to cooperate in a coalition with other service-providers.

\end{itemize}
\end{frame}



\begin{frame}
\frametitle{Some tools}

\psfig{file=gambit.eps,width=10cm}
\end{frame}


\begin{frame}
\frametitle{Summary}


\begin{itemize}
\item The aim is to make rational decisions, not guesses\vspace{0.3cm}

\item Game theory has methods to balance competing interests

\begin{itemize}
\item Aggressively (zero-sum)
\item Cooperatively (non-zero sum)\vspace{0.3cm}
\end{itemize}

\item Value is related to risk. 
\begin{itemize}
\item If we acts badly towards others, hope not to meet them again! (TIT-FOR-TAT)
\item Generous strategies tend to win in cooperative games.
\end{itemize}
\end{itemize}
\end{frame}


\end{document}